\documentclass[12pt,]{article}
\usepackage{lmodern}
\usepackage{amssymb,amsmath}
\usepackage{ifxetex,ifluatex}
\usepackage{fixltx2e} % provides \textsubscript
\ifnum 0\ifxetex 1\fi\ifluatex 1\fi=0 % if pdftex
  \usepackage[T1]{fontenc}
  \usepackage[utf8]{inputenc}
\else % if luatex or xelatex
  \ifxetex
    \usepackage{mathspec}
  \else
    \usepackage{fontspec}
  \fi
  \defaultfontfeatures{Ligatures=TeX,Scale=MatchLowercase}
\fi
% use upquote if available, for straight quotes in verbatim environments
\IfFileExists{upquote.sty}{\usepackage{upquote}}{}
% use microtype if available
\IfFileExists{microtype.sty}{%
\usepackage{microtype}
\UseMicrotypeSet[protrusion]{basicmath} % disable protrusion for tt fonts
}{}
\usepackage{hyperref}
\hypersetup{unicode=true,
            pdfborder={0 0 0},
            breaklinks=true}
\urlstyle{same}  % don't use monospace font for urls
\usepackage{biblatex}

\IfFileExists{parskip.sty}{%
\usepackage{parskip}
}{% else
\setlength{\parindent}{0pt}
\setlength{\parskip}{6pt plus 2pt minus 1pt}
}
\setlength{\emergencystretch}{3em}  % prevent overfull lines
\providecommand{\tightlist}{%
  \setlength{\itemsep}{0pt}\setlength{\parskip}{0pt}}
\setcounter{secnumdepth}{0}
% Redefines (sub)paragraphs to behave more like sections
\ifx\paragraph\undefined\else
\let\oldparagraph\paragraph
\renewcommand{\paragraph}[1]{\oldparagraph{#1}\mbox{}}
\fi
\ifx\subparagraph\undefined\else
\let\oldsubparagraph\subparagraph
\renewcommand{\subparagraph}[1]{\oldsubparagraph{#1}\mbox{}}
\fi

\date{}

\begin{document}

\subsection{Manuscript}\label{manuscript}

\subsubsection{Title}\label{title}

No Association of Male Circumcision and Sexual Risk Behaviors Found in
the Prioritized Sub-Saharan Countries After the WHO's Medical Male
Circumcision Promotion

\subsubsection{Authors}\label{authors}

Chyun-Fung Shi (corresponding author), Department of Biology, McMaster
University. Mike Lee, Department of Biology, McMaster University.
Jonathan Dushoff (corresponding author), Department of Biology, McMaster
University.

\subsection{Guidelines}\label{guidelines}

\textbf{Avoid using the backslash or brace characters}. Use a single
underscore for \emph{emphasis} and double for \textbf{bold}

\subsection{Abstract}\label{abstract}

\subsubsection{Context}\label{context}

\subsubsection{Objective}\label{objective}

\subsubsection{Design, Setting, and
Participants}\label{design-setting-and-participants}

\subsubsection{Main Outcome Measures}\label{main-outcome-measures}

\subsubsection{Results}\label{results}

\subsubsection{Conclusions}\label{conclusions}

\subsubsection{Funding}\label{funding}

\subsubsection{Keywords}\label{keywords}

male circumcision, HIV, sexual risk behaviors, risk compensation,
generalized linear mixed models (?), DHS

\subsection{Introduction}\label{introduction}

The idea of a possible connection between male circumcision (MC) and HIV
risk reduction has been suggested for a few decades
\autocites{Acle86}{CameSimo89}{Lind88}{HalpBail99}. In 2007, the World
Health Organization (WHO) and the Joint United Nation Programme on
HIV/AIDS (UNAIDS) recommended a medical male circumcision (MMC) scale-up
as part of biomedical HIV prevention interventions \autocite{WHO07a}
after the results of three randomized clinical trials (RCT)
\autocites{AuveTalj05}{BailMose07}{GrayKigo07} found MC reducing
female-to-male HIV transmission by up to 60\%. WHO subsequently
prioritized 14 eastern and southern African countries with high HIV and
low MC prevalence for a MMC scale-up, with a goal of 80\% coverage of
uncircumcised males aged 15-49 by 2015 \autocite{WHO11a}. By 2014, Kenya
and Tanzania had reached the goal with 108\% and 89\% targets
implemented, compared to nations like Namibia (6\%) and Malawi (8\%)
lagging behind \autocite{WHO15} (see table of MC progress in appendix).
Since not all the governments and medical communities were able to
satisfy MMC demands with sufficient resources and planning
\autocite{CurrNjeu11}, people aware of the MC benefit were likely to
seek the services at clinics or traditional healers where proper medical
guidelines were not likely provided, and the need of safe sex less
likely emphasized \autocite{GreeMaha13}.

There have been concerns of MMC possibly hinting sexual risk
compensation (SRC)
\autocites{AbboHabe13}{AndeCock12}{BrooEtze10}{CassHalp06}{EatoKali09}{EatoCain11}{GreeMaha13}{GrunHenn12}{GustKres11}{HeweHall12}{HogbLidd08}{KaliEato07}{L_EnLanh14}{MattCamp10}{PadiBuve08}{RiesAchi10}{SACE09}{WestAgot12}
because of the partial protection of MC. Risk compensation is a
cognitive-behavioral process searching for a balance between risk-taking
(e.g., intercourse without condoms) and potential benefits of risky
behaviours (e.g., sexual pleasure) responding to health intervention
\autocites{Chen13}{HogbLidd08}{KaliEato07}{KatzSchw02}{SheeChu01}{Unde13}{WilsGore04}.
Although most studies showed no significant sexual risk behaviour
changes
\autocites{AgotKiar07}{AuveTalj05}{BailMose07}{GrayKigo07}{GrayKigo12}{KongKigo12}{MattCamp08}{WestAgot14}{ConfirmCites}{MoreCites},
other findings showed some signs of SRC
\autocites{GrunHenn12}{KibiNans14}{KongSsek14}{MaugVenk12}{MaugGodl14}{RiesAchi10}{ZungSimb16}.
A comparison of before and after the UN's MMC recommendation in Uganda
found a significant change of reduction of condom use among circumcision
men \autocite{KibiSand16}. It was also found that risky sexual
behaviours were more common among the circumcised groups
\autocites{KibiNans14}{KibiSand16}, and the riskier behaviours were
found to associate with willingness to be circumcised after the UN's MMC
recommendation \autocite{KibiMaku15}. Qualitative studies observed mixed
messages \autocites{RiesAchi10}{GrunHenn12}: most respondents described
no behavioral changes, but some reported increased sexual risk behaviors
\autocites{GrunHenn12}{RiesAchi10}.

Here, we look at population-level survey data from ten of the fourteen
countries prioritized by the UN for the African MMC scale-up. Our
primary analysis compares the \emph{difference} in sexual risk
behaviours (SRB) between circumcised and uncircumcised men before and
after the UN's official MMC promotion. We also compared SRB after the
promotion in: uncircumcised men; men circumcised before the promotion;
and men circumcised after the promotion started. Analyzing survey data
provides a useful complement to RCT data, because it is not subject to
biases that may occur in RCTs due to intense HIV counselling and
education \autocite{MatoSsem07}.

\subsection{Research Questions}\label{research-questions}

Our main analysis focuses on changes in risk behaviour between post-2007
and pre-2008 (i.e., before and after the UN MMC recommendation). We
analyze the \emph{interaction} between survey year and circumcision
status in predicting sexual behaviour and whether the \emph{difference}
between the behavior of circumcised and uncircumcised men changes. We
called this the ``MC status'' analysis.

In addition, we also analyzed data from after 2007 to study the
differences of SRB among MC category (uncircumcised, previously
circumcised -circumcised before 2008, and newly circumcised men
-circumcised after 2007). This analysis focused on the variable-level
effect of circumcision category on risk behaviors at the overall effect
of circumcision category as the main predictor. We call this the ``MC
recency'' analysis.

\subsection{Methods}\label{methods}

\subsubsection{Data and Samples}\label{data-and-samples}

We used nationally representative surveys from the Demographic and
Health Surveys (DHS) in this study. Countries with DHS data met the
following criteria were selected: first, one of the 14 MMC prioritized
nations by WHO with national engagement in MMC programs either in
implementing or pilot stage \autocites{WHO11}{WHO11a}{WHO15} (also, see
table of MC progress in appendix); second, availability of DHS data both
before and after 2007; third, survey modules of the selected SRB data.
That resulted in ten nations: Kenya, Lesotho, Malawi, Mozambique,
Namibia, Rwanda, Tanzania, Uganda, Zambia and Zimbabwe (see TABLE in
Appendix).

Each nation has two datasets, and it summed up to 20 datasets in total.
The datasets were grouped into pre-MC surveys (i.e., pre-2007 and 2007),
and post MC surveys (i.e., after 2007). Male samples aged from 15-49
were selected in consistence with the WHO's scale-up targets
\autocite{WHO11a}. (see \autocite{tab:Status_characteristics} and
\autocites{tab:Recency_characteristics}[ Sample
Characteristics]{Recency} \textbf{{[}CF: we can put this in Appendix:
nation by varialbles, one pre-2008 and one post-2007.{]}}

The following samples were excluded from this study: never heard of
AIDS, never had sexual intercourse prior to the surveys, unaware of
their circumcision status, or had NA answers to any of the variables.
\textbf{{[}Maybe move this to the table descriptions{]}}

The sample population was not proportionally representative of the
countries because the survey sampling weights were not considered in
this study. Instead, we tried to make inferences about behaviour in the
surveyed population.

\subsubsection{Measurements and
Concepts}\label{measurements-and-concepts}

There was no agreement on how to evaluate sexual risk behaviours (SRB)
when analyzing intervention in association of HIV reduction
\autocite{Unde13}. Our selection of SRB was based on DHS data
availability and other related studies
\autocites{AuveTalj05}{BailMose07}{GrayKigo07}{GrayKigo12}{KongSsek14}.
Our main predictors were MC status, defined as circumcised
vs.~uncircumcised, in the MC status analysis, and MC category, defined
as uncircumcised, newly circumcised and previously circumcised (i.e.,
pre and post UN's MMC campaign) in the recency model. The main responses
were condom use at last sex and number of non-marital sexual partners in
the last 12 months in the status model. Number of non-martial sexual
partners in lifetime was analyzed only in the recency model because many
of the pre 2008 surveys did not include this module.

The co-variates in the two models included age, education, work status,
religion, wealth, residence (rural vs.~urban), marital status, media use
and HIV knowledge. (see
\autocites{tab:Status_characteristics}{tab:Status_Sample_Characteristics}
and \autocite[
\textcite{tab:Recency_Sample_Characteristics}]{tab:Recency_characteristics}).
Lesotho was excluded in the status model because it missed the condom
data in the pre-2007 DHS. Clusters and country were treated as random
factors in order to control for correlations between individuals from
the same geographic area and background. We also count for media slopes
in the random effect based on the presumption that media content in each
country was likely to be different.

\begin{itemize}
\tightlist
\item
  insert TABLE Sample\_Selection: sample size, method of selection (by
  per how many household), age, etc. A raw index of wealth was used and
  coded as a three-knot spline based on a priori decision, and age
  four-knot spline. Marital status was recoded into four categories that
  did not distinguish between formal and informal marriage, and religion
  also into four categories. Because Tanzania missed religion data, we
  therefore coded it separately along the code for religion. Number of
  non-marital sexual partners within the last 12 months was recoded from
  zero to three; and number of non martial sexual partners in lifetime
  from zero to six. Numbers exceeded the maximum was truncated as the
  maximum. The media use refers to the amount spend on newspapers, radio
  and TV. The basic HIV knowledge was based on three questions: ``Reduce
  chances of AIDS by always using condoms during sex,'' ``Reduce chance
  of AIDS: have 1 sex partner with no other partner,'' and ``Can a
  healthy person have AIDS.'' Both of media use and HIV knowledge were
  constructed into scores. (Please confirm if this is correct)
\end{itemize}

\subsubsection{Statistical Model}\label{statistical-model}

\_\_{[}CF: J and M, please confirm and update this section{]}

We used cumulative link mixed models (CLMMs) in the statistics package R
\autocite[Rpackage\_ordinal]{Rstats} in this study. The CLMM framework
allows us to model a binary or ordinal response variable, while treating
clusters and country as random effects.

Variable-level P values were calculated by sequentially dropping each
variable and comparing the resulting restricted models to the full
model. ``Prediction'' plots were made by calculating the effect of each
level or value of a predictor variable on the linear predictor of the
CLMM, using the model center as a reference point. Any sample with
missing data for a given variable was adaptively dropped from analyses
involving that variable.

(A sentence why this is a good/more sensible tool?)

\subsubsection{Scripts}\label{scripts}

Permission for using the DHS was authorized by USAID and is available
upon registration at The DHS program. All of the R scripts used to
analyze the data and produce the figures will be made available on the
web when the paper is published.

\subsection{Results}\label{results-1}

-figures and tables

\begin{itemize}
\item
  (insert TABLE Status\_characteristics Status Sample Characteristics) (
  We can put this in Appendix. The sociodemographic and sexual risk
  behaviour profile was presented in table XX in appendix. Can refer to
  https://www.ncbi.nlm.nih.gov/pmc/articles/PMC3626062/ table one
  format)
\item
  (insert TABLE Recency\_characteristics Recency Sample Characteristics)
  (ditto)
\item
  (insert FIGURE of condomStatus: the \emph{interaction} plot showing
  the mean effect of MC Status on condom use, and the interaction with
  DHS survey year. The relative odds ratio (ROR) is 0.83 (95\% CI,
  0.74-0.92; interaction P=?) figdrop/condomStatus\_intplots.pdf
\item
  (insert FIGURE of partnerYearStatus: the \emph{interaction} between
  survey year and circumcision status on numbers of non-martial sexual
  partners within the previous year. The ROR is 0.006 (95\% CI,
  -0.008-0.021; P=?) figdrop/partnerYearStatus\_intplots.pdf
\item
  (insert FIGURE of condomRecency: the comparison of condom use during
  the last sexual intercourse by circumcision category.
  P\textless{}0.001). Men circumcised after the UN's MMC promotion
  started were xx times more likely to use condom than uncircumcised
  men; and men circumcised before the MMC promotion were least likely to
  use condom.
\item
  (insert FIGURE of partnerYearRecency: the comparison of numbers of
  non-martial sexual partners within the previous year by circumcision
  category. P=0.142). Although men circumcised after the UN's MMC
  promotion started had more non-martial sexual partners within the
  previous year than men circumcised before the promotion and than
  uncircumcised men. The difference was not significant.
\item
  (insert FIGURE of partnerLifeRecency: the comparison of numbers of
  non-martial sexual partners in lifetime by circumcision category.
  P\textless{}0.001). Men circumcised before the UN's MMC promotion had
  an average of xx sexual partners compared to xx in men circumcised
  newly circumcised after the UN's campaign started and to xx to men not
  circumcised; and the difference was significant.
\end{itemize}

(Let's put these three recency results into one figure)

\textbf{{[}To do: Combine the above recency figure descriptions into one
P (to match one figure){]}}

\textbf{{[}To do: Combine the pages from the below recency figures
descriptions into one figure{]}}

\textbf{{[}Second page of each of these{]}}

\begin{itemize}
\tightlist
\item
  figdrop/condomRecency\_MCcat.pdf
\item
  figdrop/partnerLifeRecency\_MCcat.pdf
\item
  figdrop/partnerYearRecency\_MCcat.pdf
\end{itemize}

(need a sentence on how many samples in total and by nation or by data.
)

The results of the condom MC status model showed a significant
difference of condom use between pre-2008 and post 2007 surveys (See
Figure of interaction of condomStatus\_intPlots.pdf). The gap of condom
use between circumcised men and non circumcised men post 2007 was
significantly smaller than pre-2008 (The relative odds ratio (ROR) is
0.83 (95\% CI, 0.74-0.92; P=?)). Before the UN's MMC promotion started
in 2007, circumcised men were less likely to use condom in their last
intercourse compared to uncircumcised men, but MORE likely to use condom
after the promotion started. The likelihood of condom use among
circumcised men increased significantly (How to transfer ROR (0.83) into
words?)

A non-significant finding was found in numbers of non-marital sexual
partners within the last 12 months between men the pre-2008 data and
post 2007 data (See the interaction of prediction in Figure of
partnerYearStatus\_intPlots.pdf). The findings showed an increases of
numbers of non-marital sexual partners by year in circumcised men in the
pre-2008 surveys to post 2007 surveys, and a decrease for non
circumcised men. But the gap of the differences between circumcised men
and non circumcised men between post-2007 and pre-2008 surveys was not
significant (the relative odds ratio (ROR) is 0.006 (95\% CI,
-0.008-0.021; P=?).

The results of the three recency analyses (e.g., post-2007 surveys) were
mixed. The condom recency results showed a significant difference among
those newly circumcised, previously circumcised and non circumcised men.
The newly circumcised men were most likely to use condom, followed by
non circumcised ones. The previously circumcised men were least likely
to use condom (P\textless{}0.001. See Figure condomRecency\_MCcat.pdf).
Regarding numbers of non-marital sexual partners in lifetime, the
results showed that previously circumcised men had significantly higher
numbers, followed by the newly circumcised, then the non circumcised
ones (P\textless{}0.001. See Figure partnerLifeRecency\_MCcat.pdf). In
the case of yearly number, although newly circumcised men had most
non-marital sexual partners followed by previously circumcised ones,
then the non circumcised, the result was not significant (P=0.143. See
Figure partnerLifeRecency\_MCcat.pdf)

The variable-level significance of co-factors and the patterns of how
each risk behavior responds to these co-variates were shown in Figures
of co-variates XXX in the Appendix. Of all the co-variates, except
wealth, education and residence (Mike, can you change area to residence
on the figures? So it will match with DHS surveys), age, religion, job,
marital status, media use and knowledge were all found with significant
levels of differences in associating with all the indice of SRB in our
models (See Figures of co-variates XXX). For example, HIV knowledge was
found positively associated with condom use, and it also positively
associated with numbers of non-marital sexual partners by lifetime and
by year. It showed that the more aware of the HIV knowledge one had, the
more non-marital sexual partners they had, and, at the same time, the
more likely to use condom during sex. Similar finding was found in media
use: the more one spent on media, the more likely they would use condom
and have more sexual partners.

Because religion and location (e.g., clusters) were controlled, we
believe our findings were not biased due to background of religious and
geographical background.

(A few words on how we decided to use ``less conventional'' statistical
presentation. I.e., without relying on P values, and why. - BE BRAVE -)

\subsection{Discussion}\label{discussion}

Overall, there was no signs of SRB in association of circumcision found
in our study. The result was similar to the three main clinical trials
\autocites{AuveTalj05}{BailMose07}{GrayKigo07}, and some follow-up
prospect studies in Kenya \autocite{WestAgot14} and in Uganda
\autocites{GrayKigo12}{KongKigo12}, yet different from the others
\autocites{GrunHenn12}{KatzSchw02}{KirbSand15}{SheeChu01}{WilsGore04}{ZungSimb16}[
confirm]{to}. The cross-sectional studies did observe some significant
behaviour change {[}@{]}, for examples, in Uganda \autocite{KibiSand15}
and in South Africa \autocite{ZungSimb16} \textbf{{[}CF:\{more cites,
see who cites Katz and Wils\}{]}}

Furthermore, we found a clear increase of safe sex behaviours in condom
use during their last sexual intercourse as presented above (See Figure
of interaction of condomStatus\_intPlots.pdf). First, despite of their
circumcision status, men in the post UN's MMC campaign were more likely
to use condom than those before the campaign. Second, circumcised men
were more likely to use condom than the uncircumcised men in the
post-campaign period compared to the pre-campaign era when circumcised
men were less likely to use condom than uncircumcised men. The results
were rather consistent compared to the results of WHEN men were
circumcised (see condom recency figure): men who circumcised after the
UN's MMC promotion were more likely to use condom than those who had
already circumcised before the promotion initiated. Such an increase in
condom use was similar to findings in a prospect cohort study
\autocite{WestAgot14} in Keyna but different from the post trial follow
up in Uganda \autocite{KongSsek14} where condom use declined despite MC
status and others \autocites{ChikMaha15}{KirbSand15}{EatoCain11}
\textbf{{[}CF: confirm and update{]}. For example, traditionally
circumcised men were found more likely to have unprotect sex in xxx
\autocite{EatoCain11} }{[}CF: update{]}.

In terms of how many non-marital sexual partners men had during the
previous year, we found no clear differences by their circumcision
status (see partner year status figure) or by when they were circumcised
(see partner year recency); and the results were similar to previous
results \autocite[.]{KibiSand15}. Although there was an increase of
numbers of non-martial sexual partners in circumcised men and a decrease
in the uncircumcised in the post MMC campaign compared to the
pre-campaign time (see partner year status figure), newly circumcised
men were more likely to haev more sexual partners (see partner year
recency). In the case of numbers in lifetime, we noted that men
circumcised before the MMC promotion clearly had more sexual partners
than those newly circumcised after the campaign initiated (see
partnerLife recency). The fact that men sought for circumcision after
the MMC campaign were less likely to perform risky sex than those who
already circumcised in the comparison of total numbers of non-marital
sexual partners suggested that previously circumcised men might falsely
believe in MC protection against HIV infection and perform a riskier sex
\autocites{EatoCain11}{GrunHenn12} \_\_{[}CF: update{]}.

Most of the co-factors, including social demographic factors (i.e., Age,
religion, job, marital status), media use and HIV knowledge, also
clearly associated with the risk behaviours (see the patterns at Figure-
in appendix). In the case of media use, the findings suggested that the
more time men spent on media, the more likely they would use a condom
during sex and at the same time, the more likely to have more sexual
partners overall, the positive media association was different from a
previous finding in Kenya \autocite{MuzyLaur12}. Our findings of
increases in both condom use and numbers of non-martial sexual partners
suggested that an environment of more condom use and more causal sex.
One interpretation is that people who consume more media content tend to
have a freer style of sex in terms having more casual sexual partners
under protection (i.e., using condoms). The similar implication applies
to HIV knowledge and age: the better knowledge of HIV infection one has,
the more likely they will perform safe sex by using condom, yet more
likely to have multiple sexual partners than those with less HIV
knowledge. We also found that younger men (aged 20-30) were more likely
to use condom and to have more sexual partners within the previous years
than the others.

This was the first cross-sectional study analyzing the differences of MC
association with SRB by data before and after the UN's MMC campaign off
set and by WHEN men were circumcised. Against common anticipation of SRB
\autocites{Acle86}{CameSimo89}{Lind88}{HalpBail99}{AbboHabe13}{AndeCock12}{BrooEtze10}{CassHalp06}{EatoKali09}{EatoCain11}{GreeMaha13}{GrunHenn12}{GustKres11}{HeweHall12}{KaliEato07}{L_EnLanh14}{MattCamp10}{PadiBuve08}{RiesAchi10}{SACE09}{WestAgot12},
our findings found no clear signs of SRB. In summary, our findings
disclosed an increase of condom use with their last sexual partners
overall, especially among those newly circumcised after the UN's MMC
recommendation; and also an increase of numbers of sexual partners by
lifetime among those circumcised before the campaign, with relatively
low condom use. The results proposed that newly circumcised men were
likely more aware of the partial protection of MC compared to those who
were circumcised before the UN promotion. By identifying types of risk
compensation, we can expect a more effective and focused campaign
strategies.

\subsection{Limits and Suggestions}\label{limits-and-suggestions}

This is the first big scale analysis of cross-sectional data across SSA
on associations of MMC and SRB. We are unable to exclude all the
possible confounding factors, including.. Associations concluded from
secondary data can almost never draw causality and this limits must be
recognized.

There are two groups of targets need to be understood in order to reach
a maximum effect of MC curtailing HIV infection: women and high risk
group. This study did not tear apart SRB of high risk groups from the
general population, nor did we include women's perception of MC reducing
HIV infection and their SRB engaging a circumcised man. Their behaviours
can impact the efficiency of the goals of MC reducing HIV
\autocites{AlsaCash09}{DushPato11}{HallSing08}{WaweMaku09}{WeisHank09}.
It is also important to reach out to the traditionally circumcised men
because they were less likely to receive counselling on MC's partial
protection against HIV infection. Least but not last, analyses of media
(traditional, and social) promotion of the MMC campaign can further
optimize the accuracy of media coverage.

(A sentence on why there are difference findings in the studies, due to
statistics?)

\subsection{Conclusion}\label{conclusion}

This study of 10 nations (Kenya, Lesotho, Malawi, Mozambique, Namibia,
Rwanda, Tanzania, Uganda, Zambia and Zimbabwe) comparing data before and
after the UN's MMC campaign suggests no clear changes of SRB despite the
circumcision status and when the circumcision was received. It is noted
that the samples analyzed in this study were collected for DHS, and
received no special consultation on safe sex regarding MC's partial
protection, hence was able to mitigate some concerns about safe
education and counselling during RCTs. It is important that a safe sex
education shall reach out to the general population and not focus only
on the potential targets \autocite{EatoCain11}.

\subsection{Author Information}\label{author-information}

\subsubsection{Corresponding Author}\label{corresponding-author}

\subsubsection{Author contributions}\label{author-contributions}

\subsubsection{Financial Disclosures}\label{financial-disclosures}

None reported

\subsubsection{Funding/Support}\label{fundingsupport}

CF was funded by a grant from the James S. McDonnell foundation. Mike
\ldots{} JD holds a New Investigator award from the Canadian Institutes
of Health Research.

\subsubsection{Role of the Sponsors}\label{role-of-the-sponsors}

The James S. McDonnell foundation and the Canadian Institutes of Health
Research had no role in study design; collection, analysis, and
interpretation of data; the writing of the manuscript; or in the
decision to submit the manuscript for publication. The views expressed
herein do not necessarily represent the views of the founding bodies.

\subsubsection{Disclaimer}\label{disclaimer}

The findings and conclusions of this article are those of the authors
and do not necessarily represent the views of the funding agency.

\subsubsection{Additional Contributions}\label{additional-contributions}

Acknowledgement: Ben Bolker, Fiona Kouyoumdjian, Marta Wayne, Audrey
Patocs, David Champredon

\subsubsection{Appendix}\label{appendix}

\begin{itemize}
\tightlist
\item
  figdrop/partnerYearStatus\_isoplots.pdf
\item
  figdrop/condomStatus\_isoplots.pdf
\end{itemize}

\textbf{{[} 1st page of each of these{]}}

\begin{itemize}
\tightlist
\item
  figdrop/condomRecency\_MCcat.pdf
\item
  figdrop/partnerLifeRecency\_MCcat.pdf
\item
  figdrop/partnerYearRecency\_MCcat.pdf
\end{itemize}

\subsection{Appendix}\label{appendix-1}

\begin{itemize}
\tightlist
\item
  table of MC progress: annual numbers of voluntary medical circumcision
  in east and southern Africa,
  2008-2014/http://apps.who.int/iris/bitstream/10665/179933/1/WHO\_HIV\_2015.21\_eng.pdf
\item
  five figures of effects of co-factors.
\end{itemize}

\printbibliography

\end{document}
