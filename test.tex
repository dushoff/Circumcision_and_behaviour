\documentclass[12pt,]{article}
\usepackage{lmodern}
\usepackage{amssymb,amsmath}
\usepackage{ifxetex,ifluatex}
\usepackage{fixltx2e} % provides \textsubscript

\usepackage{hyperref}
\hypersetup{unicode=true,
            pdfborder={0 0 0},
            breaklinks=true}
\urlstyle{same}  % don't use monospace font for urls

\usepackage{longtable,booktabs}
\IfFileExists{parskip.sty}{%
\usepackage{parskip}
}{% else
\setlength{\parindent}{0pt}
\setlength{\parskip}{6pt plus 2pt minus 1pt}
}
\setlength{\emergencystretch}{3em}  % prevent overfull lines
\providecommand{\tightlist}{%
  \setlength{\itemsep}{0pt}\setlength{\parskip}{0pt}}
\setcounter{secnumdepth}{0}
% Redefines (sub)paragraphs to behave more like sections
\ifx\paragraph\undefined\else
\let\oldparagraph\paragraph
\renewcommand{\paragraph}[1]{\oldparagraph{#1}\mbox{}}
\fi
\ifx\subparagraph\undefined\else
\let\oldsubparagraph\subparagraph
\renewcommand{\subparagraph}[1]{\oldsubparagraph{#1}\mbox{}}
\fi

\bibliographystyle{plain}

\date{}

\begin{document}

\subsection{Manuscript}\label{manuscript}

\subsubsection{Title}\label{title}

No Association of Male Circumcision and Sexual Risk Behaviors Found in
the Prioritized Sub-Saharan Countries After the WHO's Medical Male
Circumcision Promotion

\subsubsection{Authors}\label{authors}

Chyun-Fung Shi (corresponding author), Department of Biology, McMaster
University. Mike Lee, Department of Biology, McMaster University.
Jonathan Dushoff, Department of Biology, McMaster
University.

\subsection{Abstract}\label{abstract}

\subsubsection{Context}\label{context}

\subsubsection{Objective}\label{objective}

\subsubsection{Design, Setting, and
Participants}\label{design-setting-and-participants}

\subsubsection{Main Outcome Measures}\label{main-outcome-measures}

\subsubsection{Results}\label{results}

\subsubsection{Conclusions}\label{conclusions}

\subsubsection{Funding}\label{funding}

\subsubsection{Keywords}\label{keywords}

male circumcision, HIV, sexual risk behaviors, risk compensation,
generalized linear mixed models (?), DHS

\subsection{Introduction}\label{introduction}

The idea of a possible connection between male circumcision (MC) and HIV
risk reduction has been suggested for a few decades
\cite{Acle86, CameSimo89, Lind88, HalpBail99}. In 2007, the World
Health Organization (WHO) and the Joint United Nation Programme on
HIV/AIDS (UNAIDS) recommended a medical male circumcision (MMC) scale-up
as part of biomedical HIV prevention interventions \cite{WHO07a}
after the results of three randomized clinical trials (RCT)
\cite{AuveTalj05, BailMose07, GrayKigo07} found MC reducing
female-to-male HIV transmission by up to 60\%. WHO subsequently
prioritized 14 eastern and southern African countries with high HIV and
low MC prevalence for a MMC scale-up, with a goal of 80\% coverage of
uncircumcised males aged 15-49 by 2015 \cite{WHO11a}. By 2014, Kenya
and Tanzania had reached the goal with 108\% and 89\% targets
implemented, compared to nations like Namibia (6\%) and Malawi (8\%)
lagging behind \cite{WHO15} (see table of MC progress in appendix).
Since not all the governments and medical communities were able to
satisfy MMC demands with sufficient resources and planning
\cite{CurrNjeu11}, people aware of the MC benefit were likely to
seek the services at clinics or traditional healers where proper medical
guidelines and counseling of safe sex were not likely provided.

\bibliography{refs}



\end{document}
